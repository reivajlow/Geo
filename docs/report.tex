\documentclass{article}

% Paquetes para configuración y formato
% \usepackage[utf8]{inputenc} % Codificación de caracteres
% \usepackage[T1]{fontenc}    % Codificación de fuente
% \usepackage[spanish]{babel} % Idioma del documento
% \usepackage{geometry}      % Configuración de márgenes
% \usepackage{graphicx}      % Para incluir imágenes
% \usepackage{hyperref}      % Para enlaces

% Configuración de márgenes
\geometry{a4paper, margin=1in}

% Título del documento
\title{Título de tu Documento}
\author{Tú Nombre}
\date{\today}

\begin{document}

\maketitle

\begin{abstract}
    Este es el resumen de tu documento. Debería ser una breve descripción del contenido y los objetivos de tu trabajo.
\end{abstract}

\section{Introducción}

Esta es la introducción de tu documento. Aquí puedes proporcionar una visión general del tema que estás abordando y explicar el propósito de tu investigación.

\section{Sección 1}

Agrega contenido a tu primera sección aquí.

\section{Sección 2}

Agrega contenido a tu segunda sección aquí.

\subsection{Subsección 2.1}

Puedes crear subsecciones para organizar tu contenido de manera efectiva.

\section{Conclusiones}

En esta sección, resume las conclusiones y hallazgos de tu trabajo.

\section{Referencias}

Incluye las referencias bibliográficas que hayas utilizado en tu investigación.

\end{document}
